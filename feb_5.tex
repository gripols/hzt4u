\lesson{1}{5 February 2025}{Natural Philosophy and Logic}

\nchapter{1}{Natural Philosophy}

Homework TODO:
\begin{itemize}
    \item Dinner Table Philosophy - Due February 10
    \item Natural Philosophers Presentation - Due February 6
    \item Reduce Feb 5 note for Philosophers
\end{itemize}

\section{Ancient Philosophers of Knowledge}

\subsection{Sophists}

Where natural philosophers looked at the world, sophists focused on people.

\begin{itemize}
    \item Athenian teachers who prioritized persuasion over truth.
    \item Truth was whatever one could argue convincingly.
    \item Politically motivated to persuade rather than seek truth.
    \item Used logical fallacies to influence others.
\end{itemize}

An example of a modern day sophist would be Donald Trump, who often persuades
without much truth. (E.g. tariffs)

\subsection{Socrates}

\begin{itemize}
    \item Opposed Sophists, seeking truth through questioning.
    \item Developed the Socratic Method—asking questions to find truth.
    \item Used dialectics (critical discussion).
    \item Wrote nothing; known through Plato’s works.
    \item Executed for "corrupting the youth."
\end{itemize}

\subsection{Plato}

\begin{itemize}
    \item Student of Socrates, believed knowledge is innate.
    \item Observations are opinions; true knowledge comes from reasoning.
    \item Theory of Forms—everything we see is an imperfect replica.
    \item Used dialectics to uncover deeper truths.
\end{itemize}

\subsection{Aristotle}

\begin{itemize}
    \item Student of Plato, but believed knowledge comes from observation.
    \item Senses are reliable; knowledge is learned, not innate.
    \item Everything has a purpose (teleology).
    \item Human purpose is to live virtuously (Golden Mean).
\end{itemize}

\pagebreak