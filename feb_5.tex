\lesson{1}{5 Febuary 2025 14:15}{Natural Philosophy and Logic}

\nchapter{1}{Natural Philosophy}

Homework TODO:
\begin{itemize}
    \item Dinner Table Philosophy - Due Febuary 10
    \item Natural Philosophers Presentation - Due Febuary 6
    \item Reduce feb 5 note for Philosophers
\end{itemize}

\section{Ancient Philosophers of Knowledge}

\subsection{Sophists}

Where natural philosophers looked at the world, sophists focused on people.

\begin{itemize}
    \item A group of Athenian teachers.
    \item Truth was whatever someone could make you believe through argument and logic.
    \item Sophists were politically motivated to persuade.
    \item Persuasion became more important than the truth.
\end{itemize}

Consider the ad "more doctors smoke Camels than any other cigarette."
This statement leaves out a fair bit of information. How many doctors smoke cigarettes?
Did they consider alternative brands, like Newports? 

A modern day Sophist could be Donald Trump. Sophists focus on persuasion with logical fallacies, 
and Donald Trump leaves out a lot of information, and has a lot of holes in his arguments.

\subsection{Socrates}

\begin{itemize}
    \item Challenged the Sophists because he believed in finding truth. Sophists were simply concerned with winning.
    \item Socratic Method -- asking questions to discover truth(s).
    \item Dialectics -- discussing an idea critically.
    \item Did not write anything down, most of what we know about Socrates is through the writings of Plato.
    \item Some say his philosophy is more about tearing down ideas than building them up.
    \item Due to his questioning, he was charged with "corrupting the youth," and was given the choice to be exiled or drink poison.
    \item He died for his beliefs, and has gone down in history as a wise and selfless seeker of truth.
\end{itemize}

\subsection{Plato}

A student of Socrates. As an "inside out thinker," he felt people had partial and unclear 
knowledge about the world around them, and that real knowledge lies within a reasoning mind. 

Plato felt observations were just opinions, and were an imperfect way of gaining knowledge.

Consider a circle. We know what it looks like in our mind, but it's very difficult to draw one freehand.
(idk expand)

Plato believed knowledge is innate; using the dialectic approach allows us to realize it. 

Plato also believed that there existed a perfect "form," and what we see are replicas.

\subsection{Aristotle}
\begin{itemize}
    \item Student of Plato, polar opposite of him.
    \item An "outside in thinker," Aristotle believed that we develop our universal concepts from observations.
    \item Felt senses are valid and accurate, and that knowledge is not innate.
    \item Everything has its purpose that is part of a larger purpose. E.g. birds have wings, so that they can fly.
    \item Human purpose is to live well and show virtues: reason, courage, honesty, and moderation in pursuing please (known as Golden Mean.)
    \item 
\end{itemize}
