\documentclass[12pt]{exam}
\newcommand{\myname}{G. Polstvin}
\newcommand{\myhwtype}{Plato's Allegory of the Cave}
\newcommand{\myhwnum}{0}
\newcommand{\myclass}{HZT4U}
\newcommand{\mylecture}{0}
\newcommand{\mysection}{Z}

% Prefix for numedquestion's
\newcommand{\questiontype}{Question}

% Use this if your "written" questions are all under one section
% For example, if the homework handout has Section 5: Written Questions
% and all questions are 5.1, 5.2, 5.3, etc. set this to 5
% Use for 0 no prefix. Redefine as needed per-question.
\newcommand{\writtensection}{0}

\usepackage{amsmath, amsfonts, amsthm, amssymb}  % Some math symbols
\usepackage{enumerate}
\usepackage{enumitem}
\usepackage{graphicx}
\usepackage{hyperref}
\usepackage[all]{xy}
\usepackage{wrapfig}
\usepackage{fancyvrb}
\usepackage[T1]{fontenc}
\usepackage{listings}

\usepackage{centernot}
\usepackage{mathtools}
\DeclarePairedDelimiter{\ceil}{\lceil}{\rceil}
\DeclarePairedDelimiter{\floor}{\lfloor}{\rfloor}
\DeclarePairedDelimiter{\card}{\vert}{\vert}

% Uncomment the following line to get Solarized-themed source listings
% You will have had to already installed the solarized-light package
% https://github.com/jez/latex-solarized
%
%\usepackage{solarized-light}

\setlength{\parindent}{0pt}
\setlength{\parskip}{5pt plus 1pt}
\pagestyle{empty}

\def\indented#1{\list{}{}\item[]}
\let\indented=\endlist

\newcounter{questionCounter}
\newcounter{partCounter}[questionCounter]

\newenvironment{namedquestion}[1][\arabic{questionCounter}]{%
    \addtocounter{questionCounter}{1}%
    \setcounter{partCounter}{0}%
    \vspace{.2in}%
        \noindent{\bf #1}%
    \vspace{0.3em} \hrule \vspace{.1in}%
}{}

\newenvironment{numedquestion}[0]{%
	\stepcounter{questionCounter}%
    \vspace{.2in}%
        \ifx\writtensection\undefined
        \noindent{\bf \questiontype \; \arabic{questionCounter}. }%
        \else
          \if\writtensection0
          \noindent{\bf \questiontype \; \arabic{questionCounter}. }%
          \else
          \noindent{\bf \questiontype \; \writtensection.\arabic{questionCounter} }%
        \fi
    \vspace{0.3em} \hrule \vspace{.1in}%
}{}

\newenvironment{alphaparts}[0]{%
  \begin{enumerate}[label=\textbf{(\alph*)}]
}{\end{enumerate}}

\newenvironment{arabicparts}[0]{%
  \begin{enumerate}[label=\textbf{\arabic{questionCounter}.\arabic*})]
}{\end{enumerate}}

\newenvironment{questionpart}[0]{%
  \item
}{}

\newcommand{\answerbox}[1]{
\begin{framed}
\vspace{#1}
\end{framed}}

\pagestyle{head}

\headrule
\header{\textbf{\myclass\ \mylecture\mysection}}%
{\textbf{\myname\ }}%
{\textbf{\myhwtype\ \myhwnum}}

\begin{document}
\thispagestyle{plain}
\begin{center}
  {\Large \myclass{} \myhwtype{} \myhwnum} \\
  \myname{} \\
  \today
\end{center}

\begin{numedquestion}
    What do you think Plato was trying to explain in the Cave?
\end{numedquestion}

It is an allegory for knowledge and enlightenment, 
where reality is not always as it seems, and people accept 
the world they are given.

I also believe that Plato had interjected some of his 
anti-democractic politics into the story.

For this to make sense, I will be comparing the allegory to The Lego Movie (2014),
as I believe both represent the same underlying messages.

\begin{numedquestion}
    What do you think each of the following symbolize?
    \begin{itemize}
        \item Chains
        \item Fire
        \item Light 
        \item Outside world
        \item Pictures on the wall 
      \end{itemize}
\end{numedquestion}

\begin{numedquestion}
    What do you think the allegory is trying to teach us about perception vs. reality?
    What is real, the cave or outside?
\end{numedquestion}

Perception is subjective, as our world is based on perceptions and biases,
You cannot and will not be able to live through the eyes of another, 
as to say their experiences shape them, but they can't shape it.

What depends on reality, 

\begin{numedquestion}
    What does this tell us about how we acquire knowledge?
\end{numedquestion}

    
\begin{numedquestion}
    Imagine if you were a released prisoner. How do you think you would react? 
    Would you want to be released from the cave? Why or why not? Explain.
\end{numedquestion}


\begin{numedquestion}
    If we leave the cave, can we ever go back?
\end{numedquestion}

Physically, sure. Mentally, no.
I think a lot of people are pretty comfortable in their ignorance.

\begin{numedquestion}
    Are we currently in "The Cave"? Can we ever know if we are in "The Cave" or not?
\end{numedquestion}

I appreciate how this question alludes to the concept of \textit{hyperreality},
as I believe it is very much worth discussing when this question comes up, and 

I find Plato is pretty clear about this.

We are all born into such a cave, as our world is based on perceptions and biases,
and the people around us impose the norms that shape our perception.

\end{document}

