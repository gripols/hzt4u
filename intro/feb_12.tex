%\lesson{4}{12 Febuary 2025}{Reading on Logic}

\section{Pages 30-33}

\subsection{Question 1}
Define and explain briefly the meanings of each of these terms: 
\textbf{logic, reasoning, argument, and inference.}

\begin{definition}[Logic]
    The inquiry which has for its object the principles of correct reasoning.

    In current usage, \textit{logic} is mainly the inquiry into deductive reasoning,
    that is, into inferences in which the conclusion follows \textit{necessarily} from the premises.
\end{definition}

\begin{definition}[Reasoning]
    The process of providing reasons in support of an idea or an action.
    A reason can be valid or invalid; convincing or unconvincing; sound or unsound; 
    yet all justify or suport a belief or action.
\end{definition}
Reasoning is important, as it is the primary method by which
philosophers support their ideas. One of the most important aspects of philosophy 
is understanding and judging the validity of reasons provided for ideas.

\begin{definition}[Argument]
    A set of propositions of which one, the conclusion, is supposed to follow 
    from the other ones, the premises. An argument is valid or invalid; correct
    or incorrect; sound or unsound; it cannot be said to be true or false.

    Of course, each of the following constiuent propositions, including the 
    conclusion and premises, can be said to be true or false.
\end{definition}

\begin{definition}[Inference]
    Mental process that occurs when we move from premises or reasons to 
    a conclusion. The process of using existing information to develop new information.
\end{definition}

\subsection{Question 2}
For each sentence below, determine which statement is an example of 
reasoning and which is not.

\begin{itemize}
    \item I enjoy the weekends and I spend time with my friends.
    \item On the weekends I spend time with my friends because I have more free time.
    \item Philosophy teaches logical thinking; it is also a way to learn to analyze arguments.
    \item Logical thinking in philosophy helps us think critically.
\end{itemize}

\pagebreak

\section{}