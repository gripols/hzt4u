%\lesson{2}{6 Febuary 2025}{Ancient Philosophers of Knowledge}

\subsection{Thales}

\begin{itemize}
    \item Greek philosopher, considered the first in Western philosophy.
    \item Believed everything originated from water, as it was the fundamental substance of all things.
    \item Though incorrect by modern standards (due to atomic theory), his ideas were revolutionary at the time.
    \item Pioneered early scientific thinking by seeking natural explanations for the world.
\end{itemize}

\subsection{Anaximander}

\begin{itemize}
    \item Greek philosopher and a student of Thales.
    \item Part of Thales' school and expanded on his teacher’s ideas.
    \item Considered the father of modern cosmology, with interests in mathematics and astronomy.
    \item Questioned what things were truly made of, arguing that water was not the fundamental element but rather a secondary state of matter.
    \item Proposed the concept of the "Apeiron" (the infinite or boundless) as the source of all things.
\end{itemize}

\subsection{Pythagoras}

\begin{itemize}
    \item Born on a Greek island but later moved to what is now Italy.
    \item A mathematician and philosopher who founded a secretive school of thought.
    \item Believed that everything in nature could be reduced to mathematical principles.
    \item His ideas influenced later mathematical and philosophical thought, though the claim that the world is purely mathematical is debated.
    \item Mathematics can describe the world, but what seems rational to one person might be irrational to another.
\end{itemize}

\subsection{Parmenides}

\begin{itemize}
    \item Greek philosopher who focused on the nature of existence.
    \item Argued that change is an illusion; reality is unchanging and eternal.
    \item Believed that knowledge must come from reason, not the senses.
    \item His ideas laid the foundation for metaphysical thought in Western philosophy.
\end{itemize}

\subsection{Zeno}

\begin{itemize}
    \item Greek philosopher from southern Italy, a student of Parmenides.
    \item Famous for his paradoxes, which challenged the nature of motion and change.
    \item Argued that motion is logically impossible, despite being observed.
    \item His paradoxes, like Achilles and the Tortoise, influenced mathematical and philosophical discussions on infinity.
\end{itemize}

\subsection{Heraclitus}

\begin{itemize}
    \item Greek philosopher who believed everything is in constant change.
    \item "You cannot step into the same river twice"—everything is in flux.
    \item Thought that even if objects appeared unchanged, they were constantly evolving over time.
    \item Distinguished between everyday experience (which seems stable) and deeper change (which happens scientifically and philosophically).
    \item His views contrast with Parmenides, who claimed reality is unchanging.
\end{itemize}

\pagebreak