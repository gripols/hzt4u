\documentclass[12pt]{article}
\usepackage{amsmath}

\title{On the Trolley Problem}
\author{G. O. Polinsky}
\date{\today}

\begin{document}
    \maketitle

    \section{Parameters of the Trolley Problem}
    A runaway trolley is moving uncontrollably down a track, which separates 
    into two paths. The reader has control of a lever that can redirect 
    the trolley, and faces a moral decision: either pull the lever, 
    moving the trolley onto a track where it will kill one person, or 
    do nothing, allowing the trolley to continue on its original path and 
    kill five people. \\

    The "fat man" introduces a condition that the reader can 
    push a large man off a bridge and onto the track, stopping 
    the trolley and saving the five people. \\
    
    \section{Philosophical Perspectives}
        
        Given this problem, two perspectives emerge.

        \subsection{The Utilitarian}
        The utilitarian believes that an action should be judged based on 
        their outcomes. With this, the value of a human life is 
        treated quantitatively, such that minimizing total harm 
        is the optimal choice. \\ 
        
        Given that the death of one person results in less overall 
        suffering than the death of five, a utilitarian will ultimately 
        pull the lever. The utilitarian did not choose this position, 
        but is otherwise obliged to reduce overall harm. \\

        \subsection{The Kantian}
        From a Kantian perspective, moral actions are judged by adherence to duty 
        and universal moral principles rather than consequences. In this view, 
        actively causing harm (such as pulling the lever or pushing the fat man) 
        is a violation of Kantian moral obligation, a \textit{Categorical
        Imperative}, which forbids using individuals as mere means to an end. 
        Therefore, removing oneself from interference aligns with their 
        moral duty, even if it results in a greater loss of life. \\

    \section{Opinion: The Trolley Problem is Stupid}
        The trolley problem explicitly excludes factors and additional moral considerations, 
        even though very act of posing the problem invites scrutiny beyond its constructed 
        constraints. The reader will inevitably consider real world complexities, 
        such as the responsibility distribution and the deception of the scenario itself. 
        Knowing this, the problem becomes less of a definitive ethical question, 
        and more so a thought experiment; something to force one to consider their 
        own logic and reasoning. \\ 

        So what is the ultimate realization, if we disregard semantics and
        variables? The fact that one has been placed in such a 
        scenario means one has already lost. Take ten for a smoke break. 

\end{document}
